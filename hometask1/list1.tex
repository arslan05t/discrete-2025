\documentclass[12pt]{article}
\usepackage[utf8]{inputenc}
\usepackage[T2A]{fontenc}
\usepackage[russian]{babel}
\usepackage{amsmath, amssymb}
\usepackage{amsthm}
\usepackage{enumitem}

\title{Толегенов Арслан\\ группа 25.Б82 \\ домашнее задание по дискре}

\begin{document}

\maketitle

\subsection*{1.2. Докажите следующие равенства в отношениях}

\subsubsection*{а) \( P(A \cap B) \subseteq P(A) \cap P(B) \)}

\textbf{Докажем \( P(A \cap B) \subseteq P(A) \cap P(B) \)}:
\[
\textbf{Пусть} \ X \in P(A \cap B) \rightarrow X \subseteq A \cap B
\]
\[
\textbf {Из}\ X \subseteq A \cap B \Rightarrow X \subseteq A \land X \subseteq B
\]
Значит, \( X \in P(A) \land X \in P(B) \), т.е. \( X \in P(A) \cap P(B) \).

\textbf{Докажем \( P(A) \cap P(B) \subseteq P(A \cap B) \)}:
\[
\textbf{Пусть}\ X \in P(A) \cap P(B). \quad \text{Тогда } X \in P(A) \land X \in P(B), \text{ т.е. } X \subseteq A  \ \land \ X \subseteq B
\]
Значит, \( X \subseteq A \cap B \), т.е. \( X \in P(A \cap B) \).

Из 1 и 2 следует равенство.

Пример строго выполнения не требуется, так как это равенство.

\subsubsection*{б) \( P(A \cup B) \supseteq P(A) \cup P(B) \)}

\textbf{Доказательство включению}:
\[
\textbf{Пусть}\ X \in P(A) \cup P(B)
\]
\[\textbf{Тогда}\ \( X \subseteq A \lor X \subseteq B \)
\]
В любом случае \( X \subseteq A \cup B \), значит \( X \in P(A \cup B) \).

\textbf{Пример строгого включения}:
\[
A = \{1\}, B = \{2\}
\]
Тогда \( A \cup B = \{1, 2\} \)
\[
P(A) = \{\emptyset, \{1\}\}
\]
\[
P(B) = \{\emptyset, \{2\}\}
\]
\[
P(A) \cup P(B) = \{\emptyset, \{1\}, \{2\}\}
\]
\[
P(A \cup B) = \{\emptyset, \{1\}, \{2\}, \{1, 2\}\}
\]

Множество \(\{1, 2\} \in P(A \cup B)\), но \(\notin P(A) \cup P(B)\). Включение строгое.

\subsubsection*{в) \( P(A \backslash B) \subseteq (P(A) \backslash P(B)) \cup \{ \emptyset \} \)}

\textbf{Доказательство включения}:

\(\textbf{Пусть}\ X \in P(A \backslash B)\) Тогда \( X \subseteq A \backslash B \)

Из \( A \backslash B \subseteq A \) следует \( X \subseteq A \), т.е. \( X \in P(A) \)

1) Если \( X = \emptyset \), то \( X \in \{\emptyset\} \), значит \( X \in (P(A) \backslash P(B)) \cup \{\emptyset\} \) 

2) Если \( X \neq \emptyset \), то \( X \subseteq A \backslash B \) означает, что \( X \) не содержит элементов из \( B \), поэтому \( X \not\subseteq B \), и \( X \notin P(B) \).

Но \( X \in P(A) \), значит \( X \in P(A) \backslash P(B) \), и тем более \( X \in (P(A) \backslash P(B)) \cup \{\emptyset\} \).

В обоих случаях отношение выполняется.

\textbf{Пример строгого включения}:
\[
A = \{1,2\}, B = \{2\}
\]
\[
A \backslash B = \{1\}
\]
\[
P(A \backslash B) = \{ \emptyset, \{1\} \}
\]
\[
P(A) = \{ \emptyset, \{1\}, \{2\}, \{1,2\} \}
\]
\[
P(B) = \{ \emptyset, \{2\} \}
\]
\[
P(A) \backslash P(B) = \{ \{1\}, \{1,2\} \}
\]
\[
(P(A) \backslash P(B)) \cup \{ \emptyset \} = \{ \emptyset, \{1\}, \{1,2\} \}
\]
Сравниваем:
\[
P(A \backslash B) = \{ \emptyset, \{1\} \}
\]
Правая часть: \(\{ \emptyset, \{1\}, \{1,2\} \}\) множество  \(\{1,2\}\) лежит в правой части, но не лежит в левой. Включение строгое.

\subsection*{1. Доказать следующее}

\subsubsection*{а) \( A \subseteq B \cap C \iff A \subseteq B \) и \( A \subseteq C \)}

\textbf{Показать в прямую сторону \( (\Rightarrow) \)}:

\[
\text{Пусть } A \subseteq B \cap C
\]

Возьмём \( \forall x \in A \). Тогда \( x \in B \cap C \), значит \( x \in B \) и \( x \in C \).

Так как \( x \) произвольный элемент \( A \), получаем \( A \subseteq B \) и \( A \subseteq C \).

\textbf{Доказательство в обратную сторону \( (\Leftarrow) \)}:
\[
\text{Пусть } A \subseteq B \text{ и } A \subseteq C
\]
Возьмём \( \forall x \in A \). Тогда \( x \in B \) (т.к. \( A \subseteq B \)) и \( x \in C \) (из \( A \subseteq C \)), значит \( x \in B \cap C \).
Следовательно \( A \subseteq B \cap C \).
\subsubsection*{б) \( A \subseteq B \backslash C \iff A \subseteq B \) и \( A \cap C = \emptyset \)}

\textbf{Доказательство в прямую сторону \( (\Rightarrow) \)}:
\[
\text{Пусть } A \subseteq B \backslash C
\]
1. Так как \( B \backslash C \subseteq B \), то \( A \subseteq B \).

2. Предположим, что \( A \cap C \neq \emptyset \), тогда \( \exists x \in A \cap C \).

Но \( x \in A \rightarrow x \in B \backslash C \rightarrow x \notin C \), противоречие с \( x \in C \).

Значит, \( A \cap C = \emptyset \).

\textbf{Доказательство в обратную сторону \( (\Leftarrow) \)}:

\[
\text{Пусть } A \subseteq B \text{ и } A \cap C = \emptyset
\]

Возьмём \( \forall x \in A \). Тогда \( x \in B \) и \( x \notin C \) (так как если \( x \in C \), то \( x \in A \cap C \), что неверно).
Значит, \( x \in B \backslash C \).
Следовательно, \( A \subseteq B \backslash C \).

\subsection*{Задача 1.9}

\subsubsection*{В Думе 1600 депутатов образовали 16000 комитетов по 80 человек в каждом. Докажите, что найдутся два комитета, имеющие не менее четырёх общих членов.}

\section*{Доказательство}

\begin{enumerate}[label=\textbf{\arabic*.}]
    \item \textbf{Подсчитаем общее число включений депутатов в комитеты} (пары «депутат–комитет»):
    \[
    16000 \times 80 = 1\,280\,000.
    \]

    \item \textbf{Среднее число комитетов на депутата:}
    \[
    \frac{1\,280\,000}{1600} = 800.
    \]

    \item \textbf{Рассмотрим тройки} $(D, A, B)$, где депутат $D$ входит в комитеты $A$ и $B$ ($A \neq B$). \\
    Число таких троек равно
    \[
    S = \sum_{i=1}^{1600} \binom{x_i}{2},
    \]
    где $x_i$ — число комитетов, в которых состоит депутат $i$.

    \item \textbf{Оценим $S$ снизу.} \\
    Функция $f(x) = \binom{x}{2}$ выпукла при $x \ge 0$. По неравенству Йенсена:
    \[
    \frac{S}{1600} \ge \binom{800}{2} = \frac{800 \cdot 799}{2} = 319\,600.
    \]
    Отсюда
    \[
    S \ge 1600 \cdot 319\,600 = 511\,360\,000.
    \]

    \item \textbf{Предположим противное:} любые два комитета имеют не более трёх общих членов. \\
    Всего пар комитетов:
    \[
    \binom{16000}{2} = \frac{16000 \cdot 15999}{2} = 127\,992\,000.
    \]
    Тогда
    \[
    S \le 3 \cdot 127\,992\,000 = 383\,976\,000.
    \]

    \item \textbf{Противоречие:}
    \[
    511\,360\,000 \le S \le 383\,976\,000
    \]
    неверно. Значит, предположение ложно.
\end{enumerate}

\vspace{0.5cm}
\noindent
\textbf{Конечный ответ:} \\
\framebox{Найдутся два комитета, имеющие не менее четырёх общих членов.}

\subsection*{Задача 1.4}

\subsubsection*{Для каждой из функций найти область значений и указать, является ли функция инъективной, сюръективной, биекцией.
\begin{enumerate}
    \item[(а)] $f : \mathbb{R} \to \mathbb{R},\ f(x) = 3x + 1$;
    \item[(б)] $f : \mathbb{R} \to \mathbb{R},\ f(x) = x^2 + 1$;
    \item[(в)] $f : \mathbb{R} \to \mathbb{R},\ f(x) = x^3 - 1$;
    \item[(г)] $f : \mathbb{R} \to \mathbb{R},\ f(x) = e^x$;
    \item[(д)] $f : \mathbb{R} \to \mathbb{R},\ f(x) = \sqrt{3x^2 + 1}$;
    \item[(е)] $f : [-\pi/2, \pi/2] \to \mathbb{R},\ f(x) = \sin x$;
    \item[(ж)] $f : [0, \pi] \to \mathbb{R},\ f(x) = \sin x$;
    \item[(з)] $f : \mathbb{R} \to [-1, 1],\ f(x) = \sin x$;
    \item[(и)] $f : \mathbb{R} \to \mathbb{R},\ f(x) = x^2 \sin x$.
\end{enumerate}}

\section*{Решение}

\begin{enumerate}
    \item[(а)] $f : \mathbb{R} \to \mathbb{R},\ f(x) = 3x + 1$ \\
    \textbf{Область значений:} $\mathbb{R}$ (линейная функция с ненулевым угловым коэффициентом). \\
    \textbf{Инъективность:} Да, $3x_1 + 1 = 3x_2 + 1 \implies x_1 = x_2$. \\
    \textbf{Сюръективность:} Да, $\forall y \in \mathbb{R}$ уравнение $3x + 1 = y$ имеет решение. \\
    \textbf{Биективность:} Да.

    \item[(б)] $f : \mathbb{R} \to \mathbb{R},\ f(x) = x^2 + 1$ \\
    \textbf{Область значений:} $[1, +\infty)$, так как $x^2 \ge 0$. \\
    \textbf{Инъективность:} Нет, $f(1) = f(-1) = 2$. \\
    \textbf{Сюръективность:} Нет, область значений не совпадает с $\mathbb{R}$. \\
    \textbf{Биективность:} Нет.

    \item[(в)] $f : \mathbb{R} \to \mathbb{R},\ f(x) = x^3 - 1$ \\
    \textbf{Область значений:} $\mathbb{R}$ (кубический многочлен неограничен и непрерывен). \\
    \textbf{Инъективность:} Да, $x_1^3 = x_2^3 \implies x_1 = x_2$. \\
    \textbf{Сюръективность:} Да. \\
    \textbf{Биективность:} Да.

    \item[(г)] $f : \mathbb{R} \to \mathbb{R},\ f(x) = e^x$ \\
    \textbf{Область значений:} $(0, +\infty)$. \\
    \textbf{Инъективность:} Да, $e^x$ строго возрастает. \\
    \textbf{Сюръективность:} Нет, $e^x > 0$ для всех $x$. \\
    \textbf{Биективность:} Нет.

    \item[(д)] $f : \mathbb{R} \to \mathbb{R},\ f(x) = \sqrt{3x^2 + 1}$ \\
    \textbf{Область значений:} $[1, +\infty)$, поскольку $3x^2 + 1 \ge 1$. \\
    \textbf{Инъективность:} Нет, $f(x) = f(-x)$. \\
    \textbf{Сюръективность:} Нет. \\
    \textbf{Биективность:} Нет.

    \item[(е)] $f : [-\pi/2, \pi/2] \to \mathbb{R},\ f(x) = \sin x$ \\
    \textbf{Область значений:} $[-1, 1]$ (синус на этом отрезке возрастает). \\
    \textbf{Инъективность:} Да, на $[-\pi/2, \pi/2]$ синус строго возрастает. \\
    \textbf{Сюръективность:} Нет, область значений только $[-1,1]$. \\
    \textbf{Биективность:} Нет.

    \item[(ж)] $f : [0, \pi] \to \mathbb{R},\ f(x) = \sin x$ \\
    \textbf{Область значений:} $[0, 1]$. \\
    \textbf{Инъективность:} Нет, $\sin x = \sin(\pi - x)$. \\
    \textbf{Сюръективность:} Нет. \\
    \textbf{Биективность:} Нет.

    \item[(з)] $f : \mathbb{R} \to [-1, 1],\ f(x) = \sin x$ \\
    \textbf{Область значений:} $[-1, 1]$ (совпадает с кодоменом). \\
    \textbf{Инъективность:} Нет, синус периодичен. \\
    \textbf{Сюръективность:} Да, все значения кодомена достигаются. \\
    \textbf{Биективность:} Нет.

    \item[(и)] $f : \mathbb{R} \to \mathbb{R},\ f(x) = x^2 \sin x$ \\
    \textbf{Область значений:} $\mathbb{R}$ (функция неограничена по модулю и меняет знак). \\
    \textbf{Инъективность:} Нет, $f(n\pi) = 0$ для всех $n \in \mathbb{Z}$. \\
    \textbf{Сюръективность:} Да. \\
    \textbf{Биективность:} Нет.
\end{enumerate}

\section*{Ответ}

\[
\begin{array}{|c|c|c|c|c|}
\hline
\text{№} & E(f) & \text{Инъективна} & \text{Сюръективна} & \text{Биективна} \\
\hline
(a) & \mathbb{R} & \text{Да} & \text{Да} & \text{Да} \\
(b) & [1, +\infty) & \text{Нет} & \text{Нет} & \text{Нет} \\
(c) & \mathbb{R} & \text{Да} & \text{Да} & \text{Да} \\
(d) & (0, +\infty) & \text{Да} & \text{Нет} & \text{Нет} \\
(e) & [1, +\infty) & \text{Нет} & \text{Нет} & \text{Нет} \\
(f) & [-1, 1] & \text{Да} & \text{Нет} & \text{Нет} \\
(g) & [0, 1] & \text{Нет} & \text{Нет} & \text{Нет} \\
(h) & [-1, 1] & \text{Нет} & \text{Да} & \text{Нет} \\
(i) & \mathbb{R} & \text{Нет} & \text{Да} & \text{Нет} \\
\hline
\end{array}
\]

\subsection*{Даны $g : A \to B$ и $f : B \to C$. Рассмотрим композицию $g\circ f : A \to C$, $(g\circ f)(x)=f(g(x))$. Определить, какие утверждения верны:
\begin{enumerate}
    \item[(а)] Если $g$ инъективна, то $g\circ f$ инъективна.
    \item[(б)] Если $f$ и $g$ сюръективны, то $g\circ f$ сюръективна.
    \item[(в)] Если $f$ и $g$ биекции, то $g\circ f$ биекция.
    \item[(г)] Если $g\circ f$ инъективна, то $f$ инъективна.
    \item[(д)] Если $g\circ f$ инъективна, то $g$ инъективна.
    \item[(е)] Если $g\circ f$ сюръективна, то $f$ сюръективна.
\end{enumerate}}

\section*{Решение задачи 1.5}

\textit{Примечание: В условии задачи композиция $g \circ f$ определена как $(g \circ f)(x) = f(g(x))$. В стандартной математической нотации это соответствует композиции $f \circ g$. В решении мы будем использовать обозначение $g \circ f$, как в условии, но подразумевая функцию $h(x) = f(g(x))$, которая отображает $A \to C$.}

\begin{enumerate}[label=(\asbuk*)]
    \item \textbf{а)Утверждение ложно.} Если $g$ инъективна, не обязательно, что $g \circ f$ инъективна.
    
    \textbf{Контрпример.}
    Пусть множества $A = \{1, 2\}$, $B = \{3, 4\}$, $C = \{5\}$.
    Определим функции:
    \begin{itemize}
        \item $g \colon A \to B$ как $g(1) = 3$, $g(2) = 4$. Функция $g$ инъективна, так как разным элементам из $A$ соответствуют разные элементы из $B$.
        \item $f \colon B \to C$ как $f(3) = 5$, $f(4) = 5$.
    \end{itemize}
    Рассмотрим композицию $(g \circ f)(x) = f(g(x))$:
    \begin{itemize}
        \item $(g \circ f)(1) = f(g(1)) = f(3) = 5$.
        \item $(g \circ f)(2) = f(g(2)) = f(4) = 5$.
    \end{itemize}
    Поскольку $(g \circ f)(1) = (g \circ f)(2)$, но $1 \neq 2$, композиция $g \circ f$ не является инъективной.

    \item \textbf{б)Утверждение верно.}
    
    \begin{proof}[Доказательство]
    Пусть функции $g \colon A \to B$ и $f \colon B \to C$ сюръективны. Мы должны доказать, что композиция $g \circ f \colon A \to C$ сюръективна.
    Это значит, что для любого элемента $c \in C$ существует элемент $a \in A$ такой, что $(g \circ f)(a) = c$.
    
    Возьмём произвольный элемент $c \in C$.
    Так как функция $f$ сюръективна, по определению существует $b \in B$ такой, что $f(b) = c$.
    Далее, так как функция $g$ сюръективна, для этого элемента $b \in B$ существует $a \in A$ такой, что $g(a) = b$.
    
    Теперь рассмотрим значение композиции в точке $a$:
    \[ (g \circ f)(a) = f(g(a)) = f(b) = c. \]
    Таким образом, для любого $c \in C$ мы нашли $a \in A$, что и доказывает сюръективность композиции $g \circ f$.
    \end{proof}

    \item \textbf{в)Утверждение верно.}
    
    \begin{proof}[Доказательство]
    Биекция — это функция, которая одновременно инъективна и сюръективна.
    
    1. \textbf{Инъективность.} Пусть $f$ и $g$ инъективны. Докажем, что $g \circ f$ инъективна.
    Пусть $(g \circ f)(a_1) = (g \circ f)(a_2)$ для некоторых $a_1, a_2 \in A$.
    По определению композиции, $f(g(a_1)) = f(g(a_2))$.
    Поскольку $f$ инъективна, из $f(y_1) = f(y_2)$ следует $y_1=y_2$. В нашем случае $g(a_1) = g(a_2)$.
    Поскольку $g$ инъективна, из $g(a_1) = g(a_2)$ следует $a_1 = a_2$.
    Следовательно, $g \circ f$ инъективна.
    
    2. \textbf{Сюръективность.} Как доказано в пункте (б), если $f$ и $g$ сюръективны, то их композиция $g \circ f$ также сюръективна.
    
    Поскольку композиция $g \circ f$ является и инъективной, и сюръективной, она является биекцией.
    \end{proof}
    

    \item \textbf{г)Утверждение ложно.} Если $g \circ f$ инъективна, не обязательно, что $f$ инъективна.
    
    \textbf{Контрпример.}
    Пусть $A = \{1\}$, $B = \{2, 3\}$, $C = \{4, 5\}$.
    Определим функции:
    \begin{itemize}
        \item $g \colon A \to B$ как $g(1) = 2$.
        \item $f \colon B \to C$ как $f(2) = 4$, $f(3) = 4$. Функция $f$ не инъективна, так как $f(2) = f(3)$, но $2 \neq 3$.
    \end{itemize}
    Композиция $g \circ f \colon A \to C$ отображает единственный элемент:
    \[ (g \circ f)(1) = f(g(1)) = f(2) = 4. \]
    Функция, определённая на множестве из одного элемента, всегда является инъективной (тривиальный случай). Таким образом, $g \circ f$ инъективна, но $f$ — нет.
    
    \item \textbf{д)Утверждение верно.}
    
    \begin{proof}[Доказательство]
    Пусть композиция $g \circ f$ инъективна. Докажем, что функция $g$ инъективна.
    Предположим, что $g(a_1) = g(a_2)$ для некоторых $a_1, a_2 \in A$. Нам нужно показать, что $a_1 = a_2$.
    
    Применим функцию $f$ к обеим частям равенства $g(a_1) = g(a_2)$:
    \[ f(g(a_1)) = f(g(a_2)). \]
    Это по определению означает:
    \[ (g \circ f)(a_1) = (g \circ f)(a_2). \]
    Поскольку по условию композиция $g \circ f$ инъективна, из этого равенства следует, что $a_1 = a_2$.
    Таким образом, мы показали, что из $g(a_1) = g(a_2)$ следует $a_1 = a_2$, что и доказывает инъективность функции $g$.
    \end{proof}
    
    \item \textbf{е)Утверждение верно.}
    
    \begin{proof}[Доказательство]
    Пусть композиция $g \circ f \colon A \to C$ сюръективна. Докажем, что функция $f \colon B \to C$ сюръективна.
    Это значит, что для любого элемента $c \in C$ существует $b \in B$ такой, что $f(b) = c$.
    
    Возьмём произвольный элемент $c \in C$.
    Так как $g \circ f$ сюръективна, существует $a \in A$ такой, что $(g \circ f)(a) = c$.
    Распишем композицию: $f(g(a)) = c$.
    
    Обозначим $b = g(a)$. Поскольку $g \colon A \to B$, элемент $b$ принадлежит множеству $B$.
    Мы получили, что для произвольного $c \in C$ существует $b \in B$ (а именно, $b=g(a)$) такой, что $f(b) = c$.
    Это по определению означает, что функция $f$ сюръективна.
    \end{proof}
\end{enumerate}

\subsubsection*{Итог}
Верными являются утверждения: \textbf{(б), (в), (д), (е)}.

\subsubsection*{Задача 1.8 У каждого из жителей города $N$ число знакомых составляет не менее 30\% населения города. Житель идёт на выборы, если баллотируется хотя бы один из его знакомых. Докажите, что можно так провести выборы мэра города $N$ из двух кандидатов, что в них примет участие не менее половины жителей.}
\section*{Доказательство}

Рассмотрим граф, где вершины — жители города, рёбра — знакомства.  
По условию, степень каждой вершины не меньше \( 0.3n \), где \( n \) — число жителей.

\subsection*{Стратегия выбора кандидатов:}

\begin{enumerate}
    \item Выберем первого кандидата \( a \) произвольно.
    \item Множество жителей, не знакомых с \( a \), обозначим \( U \). Так как степень \( a \) не меньше \( 0.3n \), то \( |U| \leq 0.7n \).
    \item Будем выбирать второго кандидата \( b \) среди всех жителей, кроме \( a \).
\end{enumerate}

\subsection*{Оценка числа проголосовавших:}

Житель \( v \) придет на выборы, если знаком хотя бы с одним из \( a \) или \( b \).  
То есть не придут только те, кто не знаком ни с \( a \), ни с \( b \).

\begin{itemize}
    \item Не знакомы с \( a \) — это множество \( U \), \( |U| \leq 0.7n \).
    \item Для \( v \in U \): чтобы \( v \) не пришел, нужно, чтобы \( b \) также не был знаком с \( v \).  
    У \( v \) степень \( \geq 0.3n \), значит, незнакомых с \( v \) — не более \( 0.7n \) (включая \( a \)).
\end{itemize}

Фиксируем \( a \). Для каждого \( v \in U \) число жителей, не знакомых с \( v \), не больше \( 0.7n \), причем \( a \) уже среди них.  
Значит, других незнакомых с \( v \) — не более \( 0.7n - 1 \).

\subsection*{Вероятностное рассуждение:}

Выберем \( b \) случайно равновероятно из \( n-1 \) жителей (кроме \( a \)).  
Для \( v \in U \):

\[
\mathbb{P}(v \text{ не знаком с } b) \leq \frac{0.7n - 1}{n-1}.
\]

При \( n \geq 10 \) это меньше \( 0.7 \).

Ожидаемое число жителей, не пришедших на выборы:

\[
\mathbb{E}[ \text{не пришли} ] \leq \sum_{v \in U} \mathbb{P}(v \text{ не знаком с } b) \leq 0.7n \cdot 0.7 = 0.49n.
\]

Следовательно, ожидаемое число пришедших:

\[
\mathbb{E}[ \text{пришли} ] \geq n - 0.49n = 0.51n.
\]

\subsection*{Заключение:}

Существует хотя бы один выбор \( b \), при котором число пришедших не меньше матожидания, то есть не меньше \( 0.51n > \frac{n}{2} \).

\vspace{0.5cm}
\noindent
\textbf{Конечный ответ:} \\
\framebox{Можно так провести выборы мэра из двух кандидатов,}\\
\framebox{что в них примет участие не менее половины жителей.}

\subsection*{Задача 1.6 Учащиеся одной школы часто собираются группами и ходят в кафе-мороженое. После такого посещения они ссорятся настолько, что никакие двое из них после этого вместе мороженое не едят. К концу года выяснилось, что в дальнейшем они могут ходить в кафе-мороженое только поодиночке. Докажите, что если число посещений было к этому времени больше 1, то оно не меньше числа учащихся в школе.}

\section*{Решение задачи 1.6}

\begin{proof}
Пусть $S = \{s_1, s_2, \dots, s_n\}$ — множество из $n$ учащихся.
Каждое посещение группы — это подмножество $G_j \subseteq S$.

Условия:
1. После посещения никакие двое из группы не могут быть вместе в другой группе. 
2. К концу года учащиеся могут ходить только поодиночке.

Это означает, что каждая пара учащихся побывала вместе ровно в одной группе.

Требуется доказать, что если $k$ — количество посещений, и $k > 1$, то $k \ge n$.

Рассмотрим следующее:

1. Для каждого ученика $s_i$, быть в группе $G_j$ означает, что он поссорился с другими участниками этой же группы.
2. Рассмотрим все пары учащихся: $\binom{n}{2} = \frac{n(n-1)}{2}$ пар.
3. Каждая пара учащихся появляется ровно в одной из $k$ групп.
4. Если предположить, что $k < n$, то в каждой группе будет меньше $n$ человек.

Теперь использовать принцип Дирихле:
- Если $k < n$, у нас не хватает «ячей» (групп) для уникальных пар. Это означает, что некоторые пары должны были бы встретиться более одного раза, что противоречит условиям задачи.

Следовательно, $k \ge n$.
\end{proof}

\end{document}